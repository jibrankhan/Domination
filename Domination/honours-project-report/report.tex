\documentclass[parskip]{cs4rep}

\usepackage{graphicx}
\usepackage{pseudocode}
\usepackage[T1]{fontenc}
\usepackage{amsmath}
\usepackage{algorithm2e}
\usepackage{array}

\begin{document}

\title{Plan Recognition in RISK}

\author{Jibran Abdus Zaahir Khan}

% to choose your degree
% please un-comment just one of the following
%\degree{Artificial Intelligence and Computer Science}
\degree{Artificial Intelligence and Software Engineering}
%\degree{Artificial Intelligence and Mathematics}
%\degree{Artificial Intelligence and Psychology }   
%\degree{Artificial Intelligence with Psychology }   
%\degree{Linguistics and Artificial Intelligence}    
%\degree{Computer Science}
%\degree{Software Engineering}
%\degree{Computer Science and Electronics}    
%\degree{Electronics and Software Engineering}    
%\degree{Computer Science and Management Science}    
%\degree{Computer Science and Mathematics}
%\degree{Computer Science and Physics}  
%\degree{Computer Science and Statistics}    

% to choose your report type
% please un-comment just one of the following
%\project{Undergraduate Dissertation} % CS&E, E&SE, AI&L
%\project{Undergraduate Thesis} % AI%Psy
\project{4th Year Project Report}

\date{\today}

\abstract{
This paper presents the design and implementation of a plan recognition agent based on an algorithm published by Christopher Geib and Robert Goldman in 2009 \cite{Geib:2009:PPR:1550966.1551246} called The Probabilistic Hostile Agent Task Tracker (PHATT). The plan recognition agents goal is to attempt to infer the unknown plan of a hostile agent from observations of their behaviour in the RISK environment.
}

\maketitle

\section*{Acknowledgements}

Michael for agreeing to supervise project, parents, family for support, C.Geib for advice and taking his time to meet with me, friends, people who helped with results

(Distinguish between academic and non-academic work.)

\tableofcontents

%\pagenumbering{arabic}

\chapter{Introduction}

\begin{flushleft}
\textit{AI has the potential to become the new driving force behind computer game innovation.}
\end{flushleft}
\begin{flushleft}
John David Funge, Artificial Intelligence for Computer Games, An Introduction
\end{flushleft}

\section{Motivations}

Artificial Intelligence for games or commonly termed 'game A.I.' has been an area of research ever since the beginning of significant work on A.I. Though techniques for game A.I. have typically come from academia it is notably different in both its scope and application.

\subsection{Difference between academic and game AI}

The primary goal of academia is to further human understanding often by solving particularly complex problems and as a result the scope of developed A.I. techniques from academia are usually more general in their application. On the other hand game A.I. is built to provide an enjoyable experience for those playing the game, usually by creating the illusion of intelligence. Game A.I. is frequently built with a singular purpose in mind, which is generally considered to be any kind of control problem in a game.

Reference - difference between academic and game A.I [AI and computer games]

\subsection{Computer Graphics previously getting more attention}

Companies in the games industry who utilize game A.I. (which today is the vast majority) tirelessly seek an edge over their competition. This edge has typically come from computer graphics, effects such as dynamic rendering, the move from two dimensional to three dimensional to keep their customers interest. 

\subsection{AI getting more attention now}

Emerging though is the idea that as users become accustomed to high quality graphics, developers will need something new to give their product an edge over their competitors. That edge will hopefully be better quality game A.I.

There has already been some examples of successful games such as F.E.A.R which Goal-Oriented Action Planning which utilized STRIPS style planning techniques[ref], the game gained much acclaim and is referenced as a good example of game A.I. by the industry and others.

With this shift in focus, there grows greater incentives and more opportunities for the continued development of more sophisticated game A.I. likely with techniques developed in academia. The application of plan recognition algorithms may provide such an opportunity. 

\subsection{Plan Recognition}

Plan recognition is the problem of inferring an agents plans from observations. [Charniak, Goldman 1992]

Research in plan recognition began in the 

\subsection{Applications of Plan Recognition}

In Nate Blaylock paper he highlighted a few of the most prominent applications of plan recognition:

User modelling, 

Multi agent interaction 

natural language processing

\subsection{Application of plan recognition algorithms to games}

Plan recognition has received attention due to its potential to provide personalized responses across the various filed of its application and game are no different.

Plan recognition can perform automated game analysis which can be used by players to improve their performance. 

(FIND AN EXAMPLE) Although not the case yet plan recognition could be potentially be applied to games, leading to responses that are personalized based on for example a players strategy preferences.

Combination with planner systems. To ideally recognise a plan and create a counter plan. The applications for such systems are endless.

\section{Artificial Intelligence and Board Games}

In 1915 Leonardo Torres y Quevedo's built a chess automaton[ref]. It was considered the worlds first computer program and arguably the beginning of Artificial Intelligence and board games.

Thirty five years later Alan Turing published a landmark paper[ref], marking what many consider to be the 'birth of Artificial Intelligence'. Soon after John McCarthy officially coined the term Artificial Intelligence at a conference in Dartmouth College[ref]. He defined it as "the science and engineering of making intelligent machines"[ref].

E.g
There existed Christopher Stracheys Game AI for the board game draughts built in 1951[ref]. As AI continued to develop so did the complexity of the A.I's seen in computer games, first AI to appear which used enemies was arcade game [ref].

Since then notable achievements such as Garry Kasparovs defeat to IBMs chess computer Deep Blue in 1997 [ref] have continued to impress the public. 

From history it seems clear that AI and its application in games are intertwined becoming a growing area of interest in both commercial and academic fields.

Games are often considered a good metric for testing the quality of an AI [Ref].. 

Some academic work done on RISK in particular. Development of an Intelligent Artificial Player for the Game of Risk[ref]. The author Michael Wolf[ref] claims RISK is generally well known but under recognised by academia.

\section{Aims}

What do we want to achieve?

An aim is a general statement of intent. It describes the direction in which the learner will go in terms of what they might learn or what the teacher/training will deliver.

To design a plan recognition agent for the board game RISK based on the PHATT algorithm developer by Geib and Goldman.

To implement a plan recognition agent for the board game RISK that can make successful predictions.

To further understanding of the complexities of performing plan recognition in the board game RISK.

\section{Objectives}

An objective is a more specific statement about what the learner should or will be able to do after the training experience.

To show plan recognition algorithms can be used successfully to predict an agent's plan in the board game RISK with an accuracy better than randomly picking a mission card.

To provide insight into the complexities of creating a plan recognition models for the RISK environment.

To provide insight into the nature of making plans in RISK. 

\section{Hypothesis}

What do you want to answer?

Are plan recognition algorithms beneficial in games? Specifically RISK?

If the probability of a mission card is the highest among all the other mission cards of an agent, then it will be the correct mission card.

The predictions of the winners will be better than that of lowers.

Why because they are an artificially derived benefit which can be used by a player to optimize their behaviour, not solely , but in conjunction with the results of a plan recognition system.

Using more data from risk environment in the computation of the likelihood of explanations, the better the prediction accuracy will be.

\section{Paper Structure}

The structure of the paper is as follows:

The following chapter presents a background to the project where plan recognition and the board game RISK are introduced. Reasons for using plan recognition in board games are then discussed.

Chapter 3 describes the methodology of the project, it is split into two sections. The first is design, in this section PHATT is introduced and the design of the plan recognition agent is detailed. The subsequent section is implementation. Code related issues such as modifications to the open source project and any relevant design concepts are discussed. 

Finally an an evaluation plan and conclusion is presented. In these, the experimental findings are presented, discussed and any conclusions derived from the experimental findings are presented.

\chapter{Background}

\section{Previous Work in Plan Recognition}

Many consider one of the earliest projects on plan recognition to have been in 1978. Having identified plan recognition as a problem in itself, Schmidt et al [ref - Schmidt Plan Recognition Problem] conducted experiments to determine whether or not people inferred the plans of other agents. From their results they created a rule based system called BELIEVER which attempted to capture the process of plan recognition. 

Three years later Cohen, Perrault and Allen identified two different types of plan recognition \textit{keyhole} and \textit{intended} [ref - cpa strategies in natural language processing]. 

They defined each as:

\begin{itemize}
\item
\textit{Keyhole plan recognition} is the recognition of an agent's plan through unobtrusive observation".
\item
\textit{Intended plan recognition} is the recognition of the plan of a cooperative agent who wishes to be understood.
\newline
\end{itemize}

In 1986 Kautz and Allen published a paper titled "Generalized Plan Recognition"[ref] which set the frame work of many plan recognition projects that followed and formed the basis of plan recognition through logic and reasoning. They defined keyhole plan recognition as involving the identification of a set of top-level goals from possible plans, which could be decomposed into related sub-goals and basic actions, thus creating an event hierarchy also known as a \textit{plan library}. 

It was Charniak and Goldman[ref - 1991 paper] who first argued that plan recognition was largely a problem of reasoning under uncertainty and that any system which did not account for uncertainty would be inadequate. They went on to propose a probabilistic rather than logic based approach to  plan recognition using a Bayesian model. Their research continues to be popular (check statement?) in many avenues of research including its application in games.

Albrecht, Zukerman and Nicholson [ref 1998 A Bayesian model for keyhole plan recognition in an adventure game] performed research on keyhole plan recognition using dynamic Bayesian networks to represent features of an adventure game. The result of their experiments they claimed "showed promise" for some domains. 

More recently Synnaeve and Bessiere [ref - RTS Bayesian model plan recognition in Starcraft 2011] published a paper of the design and implementation of a plan recognition agent that through observations of building construction in the game Starcraft, can predict the types of units a player intends to produce based on what buildings they construct.

\subsection{An Example of Plan Recognition}

We can introduce common concepts, assumptions and the process of plan recognition with an example of an agent attempting to infer the plan of another agent in a non-adversarial environment. 

Person A is cooking a meal for Person B, in other words A has a single \textit{root goal} (a state they wish to reach), such as cooked beef hamburgers. This root goal can be decomposed into a number \textit{sub-goals} of such as cook meat patties or \textit{actions} such as make meat patties  which can be further decomposed into single actions

A wanting to surprise B, will not tell B what his root goal is. B cannot know what A is thinking but B must then somehow infer what A's root goal is likely to be by \textit{observing} A cook in this way A's behaviour can be modelled as a Hidden Markov Model. Person B then models A's plan as follows. 

Person B assumes A is a rational person, and that A has a \textit{plan} to achieve this root goal. Whatever A's root goal is, it can likely be decomposed into \textit{sub-goals} such as cook hamburger meat. These sub-goals can then be broken down into basic \textit{actions} to achieve those sub-goals, such as take hamburger meat out of packet. An important point to note is that these basic actions are not limited to only being part of a sub goal.

To simplify this example, let us make some assumptions:

\begin{itemize}
\item
A believes that they can cook a meal.
\item
B can only infer the cooking plan of A by observing what A is cooking.
\item
B knows everything that A can cook.
\item
Through observing A cook, B can predict with absolute certainty what A will cook.
\item
A cannot hide any observations.
\item
A only wishes to cook one meal.
\item
A has no preferences of what to cook, in other words given a choice the probability of choosing is equally likely.
\item
One A does not change cooking plan.
\end{itemize}

Given these assumptions and B's knowledge of A's cooking abilities we can model the set of all of As plans, following the notation set by Geib and Goldman in their paper on PHATT.

\begin{figure}[h]
\centering
\includegraphics{images/example-plan-recognition}
\caption{Root goals are nodes that have no parent nodes. Sub-goals are nodes that have both a parent and children and actions are nodes that have no children. And-nodes are represented by an undirected arc across the lines connecting the parent node to its children. Or-nodes do not have this arc. Actions or goals that are dependant on a previous action are represented by an arc with an arrow.}
\label{fig:example-plan-library}
\end{figure} 

A top level or root goal is cooked vegetarian (veg) burgers, it is an and-node and this means that its children represent what needs to be done by A to accomplish that root goal. 

Children of or-nodes such as cooked meat patties represent options available to A to accomplish the goal. For example to achieve the goal cook meat patties A could either accomplish cooked chicken patties or cooked beef patties.

Arrows represent dependencies between nodes. For example to be able to accomplish the action cook beef patties, A must first perform the action Take Beef Patties out of packet.

Now A first performs the action \textit{Take Burger Bun out of Packet}, then proceeds to \textit{Toast Burger Buns}. At this point looking at the plan library we have two possible \textit{explanations} for As behaviour, cooked veg burger or cooked meat hamburger.

Each of these explanations are equally likely at this point because the actions that have occured so far could be part of either plan.

A then takes the Chicken Patties of the packet now, given our assumptions, we can conclude that A's plan is to cook beef hamburgers as A does not change his plan. In this way B has recognised A's plan thus performed the process of plan recognition.

\newpage

\section{Introduction to RISK}

Developed and released by french film director Albert Lamorisse in 1957 as La Conqu\^ete du Monde, RISK is a turn-based board game for two to six players. There exist many variations but the standard version, which this paper is concerned with, is an adversarial environment where players vie to control a fully observable board portraying a static geographical map of the Earth.

\subsection{Equipment}

\begin{figure}[h]
\centering
\includegraphics{images/risk-board}
\caption{Risk Equipment}
\label{fig:risk-equipment}
\end{figure}

The game consists of three pieces of equipment:

\begin{itemize}
\item
A board portraying a geographical map of the earth.
\item
Different coloured tokens called \textit{armies}.
\item
Two sets of regular six-sided dice.
\item
A deck of forty four cards.
\end{itemize}

\subsection{Rules}

The board is divided into forty two \textit{territories}. Each territory is a representation of a real country on Earth. These territories are partitioned into six \textit{continents} usually corresponding to their real continent grouping.

For each territory is a corresponding card, in addition to the name of that territory each card either depicts the symbol of an infantry, cavalry or artillery. By successfully occupying another player's territory in a turn, a player is then award a card and no more than one in that turn. Additionally there are two wild cards which can be substituted to represent a symbol of the players choosing.
Owning a set of three cards with the same symbols or a set of three distinct symbols gives the player the opportunity to trade the set of cards for additional armies.

Armies are placed in territories. If a player places an army in a territory, this declares that the player \textit{occupies} that territory. Players must have at least one army placed in a territory they own at all times. Players can choose to place as many additional armies as they wish in that territory.

How a player wins largely depends on the \textit{Game mode}. Game modes significantly impact the behaviour of players and is decided by players beforehand. In standard RISK they are:
\newline

\begin{tabular}{|l|p{11cm}|}
\hline 
\textbf{Game Mode} & \textbf{Description} \\ 
\hline 
Domination & Players aim to conquer a certain number of territories, \\ 
\hline 
Mission & Each player is given a single unique mission card. A mission card describes a state they must reach e.g. Occupy North America and Africa, for the rest of the game this is their \textit{root goal} which only they know. In order to win they can either complete this root goal or eliminate all other players. \\ 
\hline 
Capital Risk & Each player is given a Capital territory and to win a player must occupy all other Capital territories. \\ 
\hline
\end{tabular} 
\newline

This paper is concerned with the mission game mode \textbf{only}, therefore the following sections describe rules only for that game mode.

After an initial setup, each player's turn is split into three distinct phases which always occur in the same fixed order. These phases are:

\begin{enumerate}
\item
Reinforcement
\item
Attack
\item
Movement
\newline
\end{enumerate}

In each phase a player performs atleast one discrete \textit{action} which helps to further the players goals, thus forming a sequential task environment.

\subsubsection{Initial Setup}

The game begins with an an initial setup, it involves:

\begin{itemize}
\item
Territories being divided equally between players.
\item
Players being given a starting number of armies which is inversely proportionally to the number of players.
\item
Players distributing their starting armies over their territories.
\item
Each player being given a mission card.
\end{itemize}

\subsubsection{Reinforcement Phase}

At the start of a player's turn, they receive \textit{reinforcements} in the form of additional armies. The act of placing an army in a territory is called \textit{reinforcement}.  The number of additional armies received is based on the number of territories a player occupies and whether they occupy any continents.
\newline

\begin{table}[ht]
\centering
\begin{tabular}{|c|c|}
\hline 
\textbf{Occupied Continent} & \textbf{Number of Bonus Armies} \\ 
\hline 
Asia & 7 \\ 
\hline 
North America & 5 \\ 
\hline 
Europe & 5 \\ 
\hline 
Africa & 3 \\
\hline
Australia & 2 \\
\hline  
South America & 2 \\
\hline 
\end{tabular}
\caption{Occupied Continent Army Reinforcement Bonus}
\label{table:continent-bonus}
\end{table}

\subsubsection{Attack Phase}

Once a player has distributed their armies from the reinforcement phase, they can choose to \textit{attack}. 

Territories occupied by a player that contain more than one army can attack adjacent territories occupied by any other player. An attack consists of several \textit{battles}. The outcome of a battle is decided by means of rolling two sets of dice, thus making it a stochastic environment. One set is rolled for the \textit{defender} of the territory and the other for the \textit{attacker}. 

The number of dice each player receives for a battle is dependant on the number of armies placed in each of the players respective territories. The defender receives a die per army up to two armies. The attacker receives a die per army upto three armies not including the single army they are required to have in that territory while occupying it. 

The general rules of engagement are, dice rolls of each player are compared on a one to one basis in descending order of values. The player with the lower value at each comparison loses a single army, if the die are equal the attacker loses an army. The number of comparisons per battle are set by the number of dice the defending player has rolled. The attacking player can commence as many battles as they wish during an attack provided they have more than one army in the attacking territory.

An attack of a territory has three outcomes:

\begin{itemize}
\item
A \textit{failedAttack} by the attacker as they have only one army remaining in which case they must retreat and a \textit{successfulDefence} by the defender who retains the territory.
\item
A \textit{failedAttack} by the attacker as they choose to retreat before having only one army remaining and a \textit{successfulDefence} by the defender who retains the territory.
\item
A \textit{successfulAttack} by the attacker who occupies the territory and a \textit{failedDefence} by the defender who has no armies remaining and so loses the territory. The attacking player, leaving atleast one army behind, must then move armies from the territory they attacked from into the newly occupied territory.
\end{itemize}

A player can perform any number of attacks from any territory they own during their turn, provided they have more than one army in the territory they choose to attack from.

\subsubsection{Movement Phase}

When either the player chooses to end the attacking phase or can no longer attack because they do not occupy a territory which contains more than one army their movement phase begins.

During their movement phase a player may move armies from one territory to an adjacent territory they own, provided they leave atleast one army in the territory the armies were moved from. This action can only be done once per turn in this phase, after which the movement phase is finished.

After the movement phase has been completed the players turn ends and another players reinforcement phase begins.
\newpage

\section{Why Plan Recognition in Board Games}

Algorithmic landmark - such as the solution of the game of checkers 

In the realm of board games such as chess, there have for many years dominated a number of algorithms such as the:

Min-max Algorithm with Alpha-Beta Pruning

Many games today have a large number of alternative moves are stochastic and have hidden state attributes. "Applying traditional game tree search algorithms designed for perfect information games that act on the raw state representation is infeasible" [ref Monte Carlo Planning in RTS Games], this has been said because the search space for such games is large and that finding the best move would take an unreasonable amount of time (Find a ref for this)

Instead of a brute force method of finding the best alternative the idea of plan recognition offers another method. Use plan recognition algorithms one can weed out unlikely explanations before committing to a search of the 

Main benefit is that responses can be personalized.

This issue prompted research into the development of variants of the original algorithms that could cope but this 

Using these algorithms people like Kabanza are starting to be able to tackle certain aspects which are harder to model?

\chapter{Design}

\section{Introduction to PHATT}

PHATT was published by Christopher Geib and Robert Goldman in 2009. In their paper they presented the central realization of the PHATT approach being "that plans are executed dynamically and the actions that an agent takes at each step depend critically on the actions that the agent has previously taken". 

They went on to introduce a model of plan execution based on the notion of \textit{pending sets} of actions. They defined a pending set as the actions an agent could take based on the actions that the agent had already performed, calling it "actions that are pending execution by the agent".

In summary, plan execution according to PHATT was modelled as the following. An agent would first choose a root goal, then a set of plans to achieve that root goal. Any actions of those plans that had no pre-requisite actions would form the initial pending set of the agent. The agent would then perform an action from the initial pending set. This would result in some actions being added to the pending set and others being removed to form a new pending set. The agent would then continue to perform actions until an outcome such as the the agent concluding the root goal had been achieved.

From this model of plan execution, Geib and Goldman proposed an algorithm utilizing a Bayesian approach to perform probabilistic plan recognition. It computed \textit{Pr(g}|\textit{obs)}, the conditional probability of a goal \textit{g} given a set of observations \textit{obs}, by computing \textit{Pr(exp}|\textit{obs)}, the conditional probability of a particular explanation \textit{exp} of the likelihood of an agent having that root goal, given a set of observations \textit{obs}.

Using Bayes Rule they defined \textit{Pr(exp}|\textit{obs)} as:\newline

\centerline{
\textit{Pr(exp}|\textit{obs)} = \textit{Pr(exp} $\wedge$ \textit{obs)} / \textit{Pr(obs)}
}

They then (as other practical Bayesian systems do) exploited the equivalent formulae\newline

\centerline{
\textit{Pr($exp_0$}|\textit{obs)} = \textit{Pr($exp_0$} $\wedge$ \textit{obs)} / $\displaystyle\sum\nolimits_{i}$ \textit{Pr($exp_i$} $\wedge$ \textit{obs)}
}

This they described as the conditional probability of the explanation $exp_0$ being computed by dividing the conditional probability of that explanation by the sum of the probability mass associated with all possible explanations.

\subsection{Computing an Explanation's Probability}

To compute \textit{Pr(exp} $\wedge$ \textit{obs)} requires three probabilistic features:\newline

\begin{itemize}
\item
The probability of the root goal.
\item
The respective probabilities of choosing any sub-goals.
\item
The probabilities of picking actions from the agents pending set.\newline
\end{itemize}

The probability of an explanation is then calculated by multiplying together the probability of the root goal, the probabilities of each sub goal, and the probability of the observed actions being chosen as follows: REPHRASE\newline

\centerline{
\textit{Pr(exp} $\wedge$ \textit{obs)} = \textit{P}(\textit{goals})\textit{P}(\textit{plans}|\textit{goals})\textit{P}(\textit{obs}|\textit{exp})
}

\section{Environment Modelling}

\subsection{Root Goals}

The root goals of the RISK environment are the mission cards, these are:

\begin{itemize}
\item
Occupy Europe, Australia and one other continent.
\item
Occupy Europe, South America and one other continent.
\item
Occupy North America and Africa.
\item
Occupy North America and Australia.
\item
Occupy Asia and South America.
\item
Occupy Asia and Africa.
\item
Occupy 24 territories.
\item
Occupy 18 territories and occupy each with atleast two troops.
\item
Eliminate a player.
\newline
\end{itemize}

A subset of these where the mission involved occupying continents were chosen to be the focus of this paper. These mission cards are of two types:

\textit{Two-Continent}: Players must occupy two explicitly named continents.

\textit{Two-Continent+1}: Players must occupy two explicitly named continents and another of their choice.

\subsection{Sub-Goals}

The root goals of the RISK environment can be decomposed to form a set of sub-goals. 

For \textit{Two-Continent} root goals, the sub-goals of that root goal are occupying each continent. 

For \textit{Two-Continent+1} root goals which requires players to allow an option to occupy a continent of choice, provided they also occupy two explicitly named continents. 

These two types of root goals therefore make the occupation of any single continent a sub goal of atleast one root goal.

\subsection{Actions}

Excluding card related actions, in RISK the only actions players perform are:

\begin{itemize}
\item
Attacking a Territory.
\item
Defending a Territory.
\item
Occupying a Territory.
\item
Losing a Territory.
\item
Reinforcing a Territory.
\item
Moving armies into a Territory.
\item
Moving armies out of a Territory.
\end{itemize}

Each of these action must be modelled in a manner that contributes towards explanations of a player's behaviour.

\subsubsection{Attacking}

Though players can perform several attacks during their turn, they can only attack one territory at a time. This means that players' must choose what territory to attack and, if they wish to attack additional territories, in which order to attack provided they have adjacent territories and sufficient armies. By modelling these choices, it provides an avenue to infer a players plan.

The pending set of a players' attack actions for any turn is to either successfully or unsuccessfully attack territories they do not own that are adjacent to atleast one territory that they own which also contains atleast two armies.

If a player successfully attacks a territory then the adjacent territories that they do not own of that territory are added to their attack pending set.

If they lose a territory due to another players successful attack then the attack actions of the adjacent territories that they do not own of that territory are removed from their attack pending set for their next turn. 

The attack action thus modelled as a singular event that is either \textit{consistent} or \textit{inconsistent} with explanations of a players behaviour.

For example, given three explanations:

\begin{enumerate}
\item
Occupy North America and Australia.
\item
Occupy North America and Africa.
\item
Occupy Asia and South America.
\end{enumerate}

Successfully occupying a territory in North America would be consistent with the explanations 1 and 2 but inconsistent with 3 or in probabilistic terms the likelihood of 1 and 2 would rise whereas 3 would fall.

Having defined the outcome of attacks as either successful or unsuccessful this means that attacks can also be decomposed into the following: 

\begin{table}[ht]
\centering
\begin{tabular}{|c|c|p{8cm}|}
\hline 
\textbf{Action} & \textbf{Consistent}  & \textbf{Reasoning} \\ 
\hline 
SuccessfulAttack & Yes & Arguably the best indication of a players plan is the territories they attack and successful attacks is in itself the best outcome. \\ 
\hline 
FailedAttack & Yes & Though less significant than a successful attack, attacking a territory is indicative of a players intention to occupy that territory and therefore is a consistent action.\\ 
\hline
\end{tabular}
\caption{Modelling Attack Actions}
\label{table:attack-defend-modelling}
\end{table}

FailedAttacks will have a lower weighting than SuccessfulAttacks.

\newpage

\subsubsection{Defending}

Defending as opposed to attacking can be seen as a 'passive' action, an attack is required before a defence can occur. In that way it can be modelled rather as only inconsistent with the explanations it is directly related to and, as an act in itself, not related to those it is not.

For example, given the three explanations from the previous section. If a defence were to occur in a territory in North America, this should have minimal effect on the likelihood of explanation 3. 

Defence in the same manner as attack can be either successful or unsuccessful, therefore defence can also be decomposed into the following: 

\begin{table}[ht]
\centering
\begin{tabular}{|c|c|p{8cm}|}
\hline 
\textbf{Action} & \textbf{Consistent}  & \textbf{Reasoning} \\ 
\hline 
SuccessfulDefence & Yes & A successful defence of a territory may be purely based on chance, but is more likely when a player has invested armies in the defence of that territory and so provides a good sign that they have an plan involving that territory. \\ 
\hline 
FailedDefence & No & This is an inconsistent action because a player would not allow a territory to be lost of it is important to their plan. \\ 
\hline
\end{tabular}
\caption{Modelling Defence Actions}
\label{table:attack-defend-modelling}
\end{table}

\subsubsection{Reinforce}

The reinforce actions that a player $p$ can perform at any turn $t$ is based solely on the territories that $p$ owns during turn $t$. The pending set of any player for the reinforce actions is therefore modelled as follows. 

For each territory $T$ that $p$ owns at a certain turn $t$. In $p$'s pending set is an action to reinforce $T$. If a territory $T_{l}$ is lost by $p$ (it is attacked then occupied by another player), its corresponding reinforce action is removed from the players pending set for the players turn at $t+1$. Conversely if another territory $T_{o}$ is occupied by $p$ then a reinforce action for $T_{o}$ is added to the players pending set at turn $t$.

\subsubsection{Movement}

The pending set of movement actions of player $p$ is the set of territories a player owns that are adjacent to a territory where player $p$ has more than one army. Moving armies into a territory is modelled as a consistent action with the explanation that the territory is directly related to.

\subsection{Continent}

Each continent contains atleast two territories and so can be modelled simply as a tuple of the name of the continent and the set of territories that are contained in that continent.

$C$ = \{continentName, <$T_{1}$ ... $T_{n}$>\}

\subsection{Territory}

Each territory is modelled as an entity. When a player occupies a territory, the player gains access to a set of actions which together form that territories associated action set.

The action set of any territory is:

\begin{table}[ht]
\centering
\begin{tabular}{|c|c|}
\hline 
\textbf{Action} & \textbf{Description} \\ 
\hline 
SuccessfulAttack & Attacking and gaining the territory.\\ 
\hline 
FailedAttack & Attacking and failing to gain the territory.\\ 
\hline 
SuccessfulDefence & Retaining a territory after an attack.\\ 
\hline 
FailedDefence & Losing the territory.\\
\hline
Movement & Moving armies in to the territory.\\
\hline  
Reinforce & Placing armies in the territory.\\
\hline 
\end{tabular}
\caption{Territory Action Set}
\label{table:continent-bonus}
\end{table}

The data structure of a territory therefore is $T$ = \{ territoryName, $TA$, $N$ \} where $TA$ is a set of territory action and $N$ is a set of references to neighbouring territories.

A key concept in the design of the agent is that the pending set of a player is decided \textit{a priori}, as it is based on the territories a player owns. By combining the action sets of each territory a player owns into a single set, all the actions a player can perform can be captured.

\subsection{Player}

The term \textit{player} has been and can be used inter changeably with the term agent. A defined data structure for a player is essential to be able to separate the numerous explanations of one player from another. The data structure must contain these features:

\begin{itemize}
\item
A player ID
\item
A list of territories they own.
\item
A list of explanations.
\end{itemize}

\section{Explanations}

An explanation must be designed to contain all the necessary data required to compute its probability.

It therefore contains these features:

\begin{itemize}
\item
The explanation name.
\item
A root goal.
\item
A set of sub-goals.
\item
A set of consistent actions.
\item
A set of inconsistent actions.
\end{itemize}

One might wonder why store inconsistent actions as by definition if an action is not consistent then it is inconsistent. But this is not always the case due to the model. defence actions are not considered inconsistent to explanations that they are not directly related to, but would be if any actions not in the set of consistent actions are classified as inconsistent actions.

Explanation data structure is $E$ = \{ explanationName, RG, SG, ECA, EIA \}

RG is the root goal
SG is a list of sub goals
ECA is explanation consistent action
EIA is explanation inconsistent actions

Given the list of root goals and sub-goals, the full list of explanations is:

\begin{itemize}
\item
Occupy Europe, Australia and Africa.
\item
Occupy Europe, Australia and North America.
\item
Occupy Europe, Australia and South America.
\item
Occupy Europe, Australia and Asia.
\item
Occupy Europe, South America and Asia.
\item
Occupy Europe, South America and Africa.
\item
Occupy Europe, South America and North America.
\item
Occupy Europe, South America and Australia.
\item
Occupy North America and Africa.
\item
Occupy North America and Australia.
\item
Occupy Asia and South America.
\item
Occupy Asia and Africa.
\end{itemize}

Each of these are considered as possible explanations of a players behaviour.

\subsection{Prediction Agent Pseudo Code}

\subsubsection{Building the Set of Explanations}

During the initialisation of the map, a set of explanations must be built for the environment. This will first require a specification of root goals and sub goals for all the explanations in the environment.

Then populating each explanation with consistent and inconsistent actions can be done by the following operation: 

\begin{pseudocode}[ruled]{generateExplanationList}{-}
\begin{algorithm}[H]
\ForAll {C $\in$ CL}{

	\ForAll{E $\in$ EL}{

		\If{C \textit{isSubGoal} $of$ E}{

			\ForAll{T $\in$ C}{

				\textit{addAction}(ReinforceT) to \textit{ECA}

				\textit{addAction}(MovementT) to \textit{ECA}
				\textit{addAction}(SuccessfulDefence) to \textit{ECA}
				\textit{addAction}(FailedAttackT) to \textit{ECA}

				\textit{addAction}(FailedDefenceT) to \textit{EIA}
			}
		}
	}
}
\end{algorithm}
\end{pseudocode}

isSubGoal is an operation which checks the list of sub-goals and if the name of the current continent of loop is contained in the names of the sub goals of the current explanation then it returns true.

addAction is an operation which adds the action for the respective $T$ to a specified set.

CL is simple list of all continents for the RISK map.

EL is a list of all explanations for the RISK map.

\subsubsection{Generating Pending Sets}

Given a list of all environment explanations \textit{EL} each player is simply allocated a single instance of each explanation. Since we know that the computation of the probability of an explanation requires three features two of which we have already, we then require a function that generates all the available actions a player can perform to facilitate computing the probability of a player choosing that action.

After initializing an empty \textit{newPendingSet} which is a data structure that can only contain performable actions in the environment. The above pseudo first loops through each territory \textit{T} in the list of all the territories \textit{AT} that a player owns. For each territory a \textit{ReinforceT} action and a \textit{LoseT} action is added to the newPendingSet. 

Before proceeding onto the next territory another loop is done through the list of neighbours \textit{N} of that territory with a simple if-then-else statement. If the player owns that territory a \textit{MovementT} action is added to the newPendingSet, if not then a \textit{OccupyT} action is added to the newPendingSet, at the end the newPendingSet is returned and can be cached if necessary.

\begin{pseudocode}[ruled]{generatePendingSet}{AT}
\begin{algorithm}[H]
\textit{playerPendingSet} = $\emptyset$

\ForAll {T $\in$ AT}{

	\textit{addAction}(ReinforceT) to \textit{playerPendingSet}
	\textit{addAction}(FailedDefenceT) to \textit{playerPendingSet}
	\textit{addAction}(SuccessfulDefence) to \textit{playerPendingSet}

	\ForAll{N $\in$ T}{
			
		\eIf {playerOwnsN}{

			\textit{addAction}(MovementT) to \textit{playerPendingSet}
		}{

			\textit{addAction}(FailedAttackT) to \textit{playerPendingSet}
			\textit{addAction}(SucessfulAttackT) to \textit{playerPendingSet}
		}
	}

	\Return{newPendingSet}
}
\end{algorithm}
\end{pseudocode}

\subsubsection{Computing Action Probabilities}

Following an idea found from a paper by Geib which proposes a weighting towards consistent actions [ref - new model of plan recognition] that contribute to a plan. A method of assigning probabilities to pending set actions was deduced.

Required a simple operation that would allow weighting between actions.

\begin{pseudocode}[ruled]{ComputeBaseWeight}{weight, totalActionSetSize, consActionSetSize}
\begin{algorithm}[H]

sumWeight = weight * consActionSetSize

leftOver = 1.0 - sumWeight

base = leftOver / numActions

\Return base

\end{algorithm}
\end{pseudocode}

Retreiving the consistent actions \textit{expConsActions} of each explanation and a list of the territories a player owns one can create a set of actions of any type that a player can perform as well as a sub-set of those actions are consistent or inconsistent with any explanation. To do this is as follows:

Use the generatePendingSet operation to build a complete pending set of actions for a player.

Filter the list of all actions a player can perform to be the set of all attack actions = totalActions

Make a count of the number of actions in totalActions
Make a count of the number of consistent actions in the filtered set.

Filter from totalAttackActions any inconsistent attack actions.

Decide on a \textit{weight} you want to give actions.

\begin{pseudocode}[ruled]{computeObsProb}{weight}
\begin{algorithm}[H]
\textit{playerPendingSet} = \textit{generatePendingSet()}\newline

\textit{allExpConsActions} = set of all of explanations consistent actions
\textit{expConsActions} = \textit{filerPendingSet}(allExpConsActions, actionType) \newline

\textit{playerActionTypeSet} = \textit{filerPendingSet}(playerPendingSet, actionType)
totalActionsCount = size of playerActionTypeSet \newline

\textit{playerConsActionSet} = $\emptyset$ \newline

\ForAll{A $\in$ \textit{playerActionTypeSet}}{
			
	\If {A $\in$ \textit{expConsActions}}{

		\textit{addAction}(A) to \textit{playerConsActionSet}
	}
}
playerConsActionNum = size of playerConsActionSet\newline

base = \textit{computeBaseWeight}(weight, totalActionsCount, playerConsActionNum)\newline

conActionProb = base + weight

inConActionProb = base \newline

\Return conActionProb, inConActionProb

\end{algorithm}
\end{pseudocode}

MAKE NOTE THAT PSEUDO CODE CAN BE APPLIED TO INCONSISTENT PROBABILITIES AS EASILY.

Applying this formulae to each explanation respectively, the number computed is then multiplied by the probability of the explanation.

This formulae makes three assumptions:

That any territories including ones that only contain one army can attack. Though this is not technically true, due to the nature of RISK where players are not restricted in any way on where they may place their armies during the reinforcement phase a player can practically attack any territory they do not own but have a territory they own adjacent to it at the start of their turn.

That the likelihood of consistent attack actions are uniformly distributed.

That the likelihood of inconsistent attack actions are uniformly distributed.

\subsubsection{Computing Explanation Probabilities}

\begin{pseudocode}[ruled]{computeExpProb}{Obs}
\begin{algorithm}[H]

\If{\textbf{not} alreadyIni}{ 

	\textit{expProb} = 1.0
}

\textit{expProb} = \textit{R} * \textit{expProb}	

\textit{expProb} = \textit{computeObsProb(Obs)} * \textit{expProb}

\textit{expProb} = normalized{expProb}

\Return{expProb}
\end{algorithm}
\end{pseudocode}

The above pseudo code details the computation of an explanation. After initialising the float \textit{expProb} we multiply the explanation by the root probability. Now according to the specifications of PHATT we would normally multiply the method choice probabilities next but since each player has the full set of explanations we have eliminated the need for this term as we measure each explanation against the observations.

Using the method computeObsPro, the last loop computes the probability of choosing each observation $O$ by the player from the list of observations $OL$.

The computeObsProb method is a vitally important operation as it computes one of the three probabilistic features required by the algorithm.

Normalizing the explanation probability at the end of each operation is also vital as this prevents the probability of any explanation from dropping to zero at a near exponential rate. 

\subsubsection{Example of Operation}

These concepts can be tied together with an arbitrary example.

As mission cards are handed out to players' at random, the prior probability of each root goal is 1/\textit{n} where \textit{n} is the number of mission cards.

As we know the probability of being a mission card is uniform and though there are six mission cards, two of those have four derivatives resulting in 12 explanations of a players behaviour. Therefore the probability of a single explanation is 1/12 by doing this sub goal probabilities are removed as at all times we consider the full set of explanations.

Introducing some other assumptions:

The probability of a player choosing to attack any territory is uniform.
The probability of a player choosing to occupy a continent as a sub-goal is uniform.
The RISK map has every territory connected a with every other territory.

\subsection{Conceptual Issues}

Main change is that PHATT works in an action space whereas I have had to adapt the implementation to represent the state of the 'risk world' for the algorithm to operate.

Another difference is that I do not store or manipulated derivative trees as the original paper includes in the pseudo code it presents, this is due to hard coding derivative trees into explanations.

Another difference is I have removed the need for sub goal probabilites because of working with the full derivative tree.

Normalising after each observation computation rather than only when a probability needs to be computed.

Using the constraints in the environment initially the model was based on the plan library proposed in the PHATT paper. This was not good therefore utilized the constraints of countries to narrow the number of possible explanations and actions.

Initially the movement model was based on the attack model where the smaller the number of actions the bigger the weighting should be, this is an incorrect model as the number of territories a player owns increases as a player gains more consistent movement and reinforce actions as they continue to be successful in their goal.

\subsubsection{Probability Assignment}

Calibrating Mechanisms of algorithm
Uniform distribution
Simple weighting based on reasoning about implications of an action to an explanation
Weighting based on number of consistent and inconsistent actions available to a player when an action is performed.

Manual Simple Weighting
Positively weighted probabilities towards consistent actions and negatively towards inconsistent actions

Finding
Equal Distribution does not work very well for simple explanations

so made up:

Weighted Distribution based on consistent and inconsistent actions

Though question is How to assign weights to consistent and inconsistent actions?

Total number of actions are counted, and are separated into inconsistent actions and consistent actions and a weight is split among inconsistent actions and consistent actions. Problem is that it seems this system is too sensitive.

Because of sensitivity had to make up a more evenly weighted automatic system.

\subsection{Summary}

The summary for this section is

\chapter{Implementation}

\section{Modifications to Open Source Project}

Initialised plan recognition agent and processing class within 

Initialisation of plan recognition agent and processing class in risk game.

Code that injected events into processing which passed it to the plan recognition agent in various places namely:

\begin{itemize}
\item
Player Initialisation
\item
Player Removal
\item
Territory Occupation
\item
Attacks
\item
Army Movements
\end{itemize}

\section{System Construction}

The architecture of the plan recognition agent was built around an existing open source RISK game developed by Yura Mamyrin (verify name) and (anyone else) in (year?)[ref - to website?].

Initially the architecture was designed as a distributed system where the system would store its own subset of data structures from the RISK environment.

Agent data was generated such as countries they owned.
Agent explanations computed by the plan recognition agent based on events from the environment.

The reason for this choice was the simplicity of its implementation at the time.

(Diagram of data structure of local copy)

Adding data to this hard coded data structure proved to be overly time consuming and therefore was changed to an automated system of the generation of a data structure allowing the plan recognition agent to work on any RISK map that could be successfully played in the open source game.

Through the modelling process the realization that I could utilize the data structures of the developer rather than storing a copy within the plan recognition agent advantage is less overhead, disadvantage is that this makes the agent more dependant on the methods of the developers implementations.

 was could use mirrored those that had already been defined by the developers of the open source game.

\section{System Architecture Concepts}

\subsection{Event Driven System}

Plan recognition agent receives event objects that are fired from the previously mentioned modifications to the original source code. These event objects contain important information from the environment which the plan recognition agent uses to compute the likelihood of explanations.

\subsection{Component Based Class Architecture}

Classes are modelled as an aggregation of components allowing

\subsection{Google Guava Libraries}

Extensive use of the freely available Google Guava libraries to allow the prediction agent to operate concurrently with the game as well as more efficiently in its operations.

\subsection{Replaying Games}

Used an inbuilt tool by the developer of saying "Debug Logs" which could be used to reproduce a game exactly, this allows the algorithm to be tested with different parameters on the exact same game thus ruling out any environmental difference when analysing data.

\section{Summary}

\chapter{Evaluation}

Two questions to answer, 

1) Does it work?

It performs better than randomly guessing but still at a low probability

2) If not why?

Nature of game, simplicity of model.

Three types of games to test the algorithm in:

\section{Experiments}

\subsection{AI Play}

The open source project has built in AI to play for the mission game mode. This provided an opportunity for the collection of a large number of data samples to test the performance of the plan recognition agent.

\subsection{Constrained Play}

Where players are asked only to perform actions that are consistent with their root goal.

This is defined as given a situation a player only:

Chooses to attack a territory that is directly consistent with their plan or a territory that is on the shortest route to a territory that is directly consistent with their plan.

Reinforces a territory that is directly consistent with their plan or a territory that is on the shortest route to a territory that is directly consistent with their plan.

Moves armies to a territory that is directly consistent with their plan or a territory that is on the shortest route to a territory that is directly consistent with their plan.

Why did I choose to do this?

To establish that the algorithm works given that players behave only a consistent manner with the mission they have been given.

\subsection{Free Play}

Where players are asked to play as normal.

\section{Experimental Format}

Explained the rules then

There are two ways to win in mission RISK, one is to eliminate all the other players, the other is to complete your mission.

Explained to players the rules of the game 

\section{Data Collection}

How did I get the data?

After the game is finished I automatically generate a log file that allows me to replay a game exactly and a csv file recording significant features such as:

a players actual mission
player explanation probabilities over time
which player won and lost

The data collection for AI play could be automated using a program called Autohotkey which allowed the creation of macros which automated the data collection process. 

Then used python scripts to analyse the csv files.

Used a tool created by developers to replay games in-order to be able to re-test the algorithm with changes in modelling/calibration.

Using an system built by the original developers was able to easily reproduce a game if necessary for the purposes of evaluating the algorithm under different conditions.

Questions to answer:

What did I tell players?

Explained to players the rules of the game and that there were two primary methods of winning in mission risk either by eliminating their opponents or by completing their mission card player were instructed to "Play to Win"

\subsection{Experimental De-briefing}

At the end of the experiment I asked players to give a short high-level summary of their plan, this included when their plan

asked players to give a summary of what they had done in the game, recorded this down to use for comparison purposes at a later point.

\section{Experimental Findings}

Sampling of probability needs to be more frequent rather than just at end of term as many actions occur between turns and the game may end in the middle of a lot of actions during one turn. Players do a lot of significant things in one turn.

Players who spend a game fighting over a single continent raise the probability of all explanations associated with that continent and the result is a prediction of several explanations being equally likely. This also applies to fighting over two continents that are part of a three continent explanation.

Misclassification by Association - Explanations appear likely because players do things related to them even though they haven't done anything in one of the continents e.g Europe SA, Asia appears likely even though there has been occupation of territories in Europe because of lots of activity in SA and Asia. This issue is slightly negated by how I have modelled the attack actions probability contribution to explanations.

Main questions to answer.

Whether is converges?

It does converge.

How fast it converges?

Converges quite quickly after the attacks begin - use data from average data over time.

Players don't ignore their mission card like I first thought, as long as the mission seems easier to complete than eliminating all other players then players will try achieve their mission.

\section{Outcomes}

In evaluation DO NOT JUST RELY ON MEASUREMENTS! DISCUSS OUTCOMES! MAKE INFERENCES FROM DATA!

THE WHOLE PURPOSE OF ACADEMIA IS TO LEARN, SHOW THAT YOU HAVE SOMETHING TO CONTRIBUTE TO THAT AIM TO LEARN!

Like what did you find hard? What did you do that you found easy? What did you do that you found hard? 

Would you recommend doing something, what would you recommend not doing?

TALK ABOUT THE OUTCOMES OF YOUR WORK!

NATURE OF GAME PROBLEM

Inherent issues with PHATT is unable to deal with deception[ref] and often players must perform actions unrelated to a plan to be able to complete their plan e.g survive, have to conquer other non-relevant continents, this confuses the algorithm.

The problem of prediction by association, this is where an explanation which involves a continent that a player has not done anything at all to becoming likely because the explanation contains two other continents that the algorithm has observed many things happen to.

How to decide weighting of 

How to deal with this problem?

Using the plan recognition software given a programs a plan to detect how well a program performs a plan. 

Using genetic algorithms to optimize this by looking at the numbers returned by the plan recognition algorithm and choosing the best configuration that survived and won.

\section{Criticism}

Essentially attack has been modelled following PHATT but the rest of the actions are simple given a weighting of 0.98 for inconsistent and 1.0 for consistent. Thus attack has been modelled as the most significant action in the environment and this we can intuitively tell is not the case.

\chapter{Conclusions}

\section{Future Work}

The application of more sophisticated computation models based on the idea of generating a pending set from the state of the world then performing calculations with that pending set.

Only attack actions have been modelled in a matter 

Probability model changed to proportion of consistent actions * 0.1 
0.9 - proportion for consistent probability
1 - proportion for inconsistent probability

MODEL AUGMENTATION

CRITICISM - ASSUMPTIONS ARE NOT ALWAYS TRUE BECAUSE OF NON-DETERMINISTIC NATURE OF BATTLES! AND OF MULTIPLE ATTACKS PER TURN FOR PLAYERS

MODEL PROBLEM - MODEL HAS BEEN BUILT NOT TAKING THE EFFECTS OF OTHER PLAYERS INTO ACCOUNT EFFECTIVELY

MODEL PROBLEM - PLAYERS MAY BE IN A SITUATION WHERE THEY CANT PERFORM A CONSISTENT ACTION

MODEL PROBLEM - DOES NOT TAKE EFFECT OF TRADING CARDS INTO ACCOUNT

MODEL PROBLEM - DID NOT MODEL MOVING ARMIES INTO A TERRITORY

MODEL PROBLEM - DID NOT MODEL BASED ON IMPORTANCE OF TERRITORIES E.G DID NOT TAKE INTO ACCOUNT CONTINENT ENTRY TERRITORIES

MODEL PROBLEM - ATTACK, TREATS ANY NON OWNED CONNECTED TERRITORY AS ATTACKABLE, GOES ON ASSUMPTION THAT PLAYER HAS AT LEAST TWO ARMIES IN EVERY TERRITORY

MODEL PROBLEM - Talk about other players using a players own territory as a blocking mechanism to prevent another player attacking.

PROBLEM - DUE TO NORMALIZING OTHER EXPLANATIONS BECOME MORE LIKELY AS A BI-PRODUCT, TRIED TO AVOID THIS BY MAKING PROBABILITIES MULTIPLICATIONS MINIMAL.

PROBLEM - MODEL NEGLECTS THE MULTI-ATTACK NATURE OF THE GAME, SUCCESSFUL DEFENCE MODEL REALLY ONLY APPLIES FOR SINGLE ATTACKS PER TURN

MODEL IS TOO LOP SIDED TOWARDS ATTACK, SHOULD TAKE INTO ACCOUNT WHAT TERRITORIES PLAYERS OWN AT THE END OF A TURN.

Augmentation possibility - look at where player has already placed armies and figure out a probability based on what they placing tells you.

CRITICISM, DO NO COMPUTATION OF NEW EXPLANATIONS AS AN INSTANCE OF THEM ARE ASSIGNED AT THE START. MEANING ACCORDING TO PHATT A LIST OF PENDING SETS SHOULD BE SAVED AND A LIST OF ACTIONS AN AGENT HAS TAKEN, BUT I HAVE NOT DONE THAT.

A platform has been setup for further improvements of the model.

IMPROVEMENTS TO MODEL

CAPTURING THE CONNECTIONS BETWEEN CONTINENTS

MAKING A TIERED WEIGHTING SYSTEM FOR THE CURRENT METHOD OF COMPUTING PROBABILITIES

E.G 

0.1 WEIGHT FOR MORE THAN HALF THE NUMBER OF CONSISTENT ATTACK ACTIONS IN SET OF ALL ATTACK ACTIONS
0.2 WEIGHT FOR HALF THE NUMBER OF CONSISTENT ATTACK ACTIONS IN SET OF ALL ATTACK ACTIONS
0.3 WEIGHT FOR 1/4 THE NUMBER OF CONSISTENT ATTACK ACTIONS IN SET OF ALL ATTACK ACTIONS
0.4 WEIGHT FOR 1/10 THE NUMBER OF CONSISTENT ATTACK ACTIONS IN SET OF ALL ATTACK ACTIONS

\section{Final}

the benefit of such systems would be to such an extent that it would be considering an unfair advantage if used in events such as official tournaments.

\chapter{Appendix}

\section{Class Diagram}

Write up class descriptions with a diagram here

\section{Experimental Results}

Table of correct predictions and incorrect predictions in each game type.

Table of correctness (an percentage of correct predictions by end of game for all data)

Graph of the line of the correct explanation for each agent in a single graph.

GRAPHS to make
Probabilities of each explanation over rounds, a graph for average correctness for each game type.
How to make this graph meaningful?

% use the following and \cite{} as above if you use bibtex
% otherwise generate bibtem entries
\bibliographystyle{plain}
\bibliography{mybibfile}

\end{document}

